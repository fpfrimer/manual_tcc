\documentclass[a4paper, 12pt]{article}

% Pacotes:
\usepackage[brazilian]{babel}
\usepackage[utf8]{inputenc}
\usepackage[T1]{fontenc}
\usepackage[margin=2cm]{geometry}
\usepackage{indentfirst}
\usepackage[colorlinks=true, urlcolor=blue]{hyperref}

% Comandos:
\newcommand{\ordm}{\textsuperscript{\underline{o}}}
\newcommand{\ordf}{\textsuperscript{\underline{a}}}
\newcommand{\utf}{Universidade Tecnológica Federal do Paraná - Câmpus Toledo}
\newcommand{\tcc}{Trabalho de Conclusão de Curso}

\title{Procedimentos de TCC2 - 2024/2\\\textbf{Engenharia Eletrônica}}
\date{***}
\author{Universidade Tecnológica Federal do Paraná - Câmpus Toledo\\Coordenação do curso de Engenharia Eletrônica}


% Atualizar as datas a cada semestre:
\newcommand{\startdate}{01/10/2024}		% Data de início do semestre
\newcommand{\agendadate}{28/06/2025}	% Prazo para desenvolvimento e escrita do TCC
\newcommand{\bancadate}{05/07/2025}		% Prazo para realização das Bancas
\newcommand{\finaldate}{11/07/2025}		% Prazo para a entrega da versão final
\newcommand{\seiprocess}{23064.019011/2025-81} % Processo para as ATAs

\begin{document}
    \maketitle
    
    \section{Modelo de TCC}
    Os trabalhos devem seguir os modelos postados na página da biblioteca \href{http://portal.utfpr.edu.br/biblioteca/trabalhos-academicos}{(clique aqui para acessar)}. Existem modelos em docx e \LaTeX, onde é possível escolher a licença do trabalho (verifique a \href{https://sei.utfpr.edu.br/sei/publicacoes/controlador_publicacoes.php?acao=publicacao_visualizar&id_documento=2615190&id_orgao_publicacao=0}{Resolução Conjunta COPPG/COGEP n\textordmasculine{} 01/2021}, de 10 de novembro de 2021, que dispõe sobre a Política de licenciamento das versões finais dos Trabalhos de Conclusão de Curso da Graduação (TCC) e dos Programas de Pós-Graduação Stricto Sensu (dissertações e teses), bem como dos produtos educacionais e tecnológicos a elas vinculados, produzidas no âmbito da Universidade Tecnológica Federal do Paraná).

	No entanto, no caso da escrita em \LaTeX{}, recomenda-se a utilização do modelo do curso devido a facilidade de uso e maior compatibilidade com pacotes atualizados. O modelo do curso pode ser baixado em \url{https://github.com/coele-td/TCCmodelv3}.
    
    \section{Cronograma}
    \label{sec:CRO}
    
    Os procedimentos para a disciplina de TCC2 seguirão o seguinte cronograma:
    \begin{itemize}
    	\item \textbf{Etapa 1} - Desenvolvimento do TCC: até \agendadate;
    	\item \textbf{Etapa 2} - Agendamento e realização das bancas: até \bancadate;
    	\item \textbf{Etapa 3} - Entrega da versão final: até \finaldate.    	
    \end{itemize}

	A entrega das documentações solicitadas é requisito para a aprovação em TCC2.

	\section{Planilha de informações}
	\label{sec:pla}
	Trata-se de uma planilha compartilhada onde os orientadores  devem inserir as informações solicitadas em cada etapa do cronograma. As informações necessárias são descritas nas próximas seções. A planilha possui acesso restrito. Portanto, orientadores devem solicitar acesso assim que acessarem o link pela primeira vez.
	
	\textbf{Atenção}: o acesso é concedido em até 5 dias úteis após a solicitação.
	
	O link para editar a planilha pode ser acessado \href{https://docs.google.com/spreadsheets/d/1kvA48JsgkLEdtAHvRgQ_IDVJ_20ch6-fIN_75WXXCfA/edit?usp=sharing}{clicando aqui}.
	
	A planilha pode ser visualizada \href{https://docs.google.com/spreadsheets/d/e/2PACX-1vRBVzCBFvFwd0x0fgkgbf2qIa7ubVvVlpGffdX8YjB73VHXoW492Vf9BC0-FdDoz8Xm2ngQCqe_pHM-/pubhtml}{clicando aqui}.

	Toda edição feita na planilha é monitorada pelo PRATCC.

	\section{Etapa 1 - Desenvolvimento do TCC}
	\label{sec:et1}
	Etapa onde os candidatos desenvolvem e escrevem o texto de TCC2. Recomenda-se uma reunião de orientação por semana. \textbf{Nesta etapa é necessário completar os dados da Etapa 1 na planilha de informações} (veja a \autoref{sec:pla}).

    \section{Etapa 2 - Agendamento e realização das bancas}
	\label{sec:et2}

	Período onde as bancas são agendas pelo orientador e as bancas são realizadas. Nesse período, deve-se preencher a seção \textbf{Etapa 2} da planilha de informações.
   	
	\subsection{Agendamento da defesa de TCC2}

    A banca de TCC2 pode ser realizada a qualquer momento entre os dias \textbf{\startdate{} e \bancadate{}}, sendo de responsabilidade do orientador a escolha de dois avaliadores. A critério do Orientador, um dos membros pode ser externo ao quadro efetivo de docentes da UTFPR, desde que atue em área relevante para o trabalho avaliado. Também é de responsabilidade do orientador agendar o dia, horário e local de apresentação. O local de apresentação deve ser reservado pelo orientador.
    
    Para formalizar a apresentação de TCC2 o orientador deverá informar o agendamento de Banca de TCC2 para o professor responsável pela atividade de TCC (PRATCC) por meio da planilha de informações (veja a \autoref{sec:pla}), na região indicada como Etapa 2.
    
    A planilha deverá ser preenchido com \textbf{antecedência mínima de 10 dias antes da apresentação} para que a banca tenha tempo hábil para corrigir o texto.    
		
    Observações importantes:
    
    \begin{itemize}
    	\item A entrega do texto para a Banca deve ser feita com antecedência mínima de 10 (dez) dias antes da defesa;
    	\item Os discentes ou os orientadores são responsáveis por entregar a cópia digital ou impressa do texto de TCC para a banca;
    	%\item A data de apresentação informada no formulário 9 devem ser previamente agendada com a banca pelo(a) Orientador(a);    	
    	\item Mudanças relativas à data de apresentação e a composição dos membros da banca deverão ser informadas e justificadas pelo orientador ao PRATCC pelo e-mail douglascoutinho@utfpr.edu.br;
    	\item O período de agendamento para os trabalhos de TCC2 está exposto na Seção \ref{sec:CRO};
    \end{itemize}

	\section{Assinaturas para avaliadores externos}
	
	Assinatura externa é aquela feita por pessoas de fora da instituição. No caso do TCC, essas assinaturas são de membros externos da banca. Para poder assinar a documentação os membros externos devem estar cadastrados no SEI. O cadastramento deve ser feito \textbf{pelo orientador}, que deverá entrar em contato com um facilitador do SEI no campus. Pergunte ao facilitador do SEI no campus Toledo \href{http://portal.utfpr.edu.br/servidores/servicos-servidor/sei/facilitadores}{(clique aqui para a lista)} quais procedimentos e dados são necessários para cadastrar o(a) avaliador(a).

	\subsection{O seminário de TCC2}
	
	O(A) presidente da banca é o orientador ou coorientador, que deverá coordenar as etapas de apresentação: defesa, arguição e deliberação. 
	
	Os candidatos terão 15 min para realizar a defesa. Logo após, cada membro da banca terá até 15 minutos para apresentar suas correções e questionamentos. Por fim, os membros da banca devem se reunir de forma reservada para deliberar o resultado. O rito da banca deve ter duração máxima de 1 h.
	
	

	\subsection{Critérios de avaliação}

	Durante a deliberação, o orientador deve avaliar de acordo com os seguintes critérios:

	\begin{itemize}
		\item Comprometimento com as determinações da orientação e participação das reuniões (peso de 15\%);
		\item Demonstrou interesse pelo aprendizado e pelo projeto (peso de 15\%);
		\item Demonstrou dedicação ao escrever as versões entregues para a revisão (peso de 15\%);
		\item Demonstrou autonomia para encontrar soluções para o projeto (peso de 15\%);
		\item Qualidade geral do texto final (normas da ABNT, clareza do texto, ortografia, coesão, entre outros) (peso de 40\%);
	\end{itemize}

	Durante a deliberação, os membros devem avaliar de acordo com os seguintes critérios:

	\begin{itemize}
		\item Demonstrou conhecimento sobre o assunto durante a apresentação (peso de 20\%);
		\item Qualidade da Discussão do Resultados e conclusões (peso de 25\%);
		\item Descreve de forma clara a metodologia / A metodologia utilizada foi adequada (peso de 25\%);
		\item Qualidade geral do texto final (normas da ABNT, clareza do texto, ortografia, coesão, entre outros) (peso de 30\%);
	\end{itemize}

	Esse critérios deverão ser divulgados pelo orientador ou candidatos no ato de entrega do texto de TCC1. A nota deve ser calculada de acordo com a \href{https://sei.utfpr.edu.br/sei/publicacoes/controlador_publicacoes.php?acao=publicacao_visualizar&id_documento=3053252&id_orgao_publicacao=0}{instrução Normativa COELE-TD/UTFPR nº 5}, de  13 de junho de 2022.

	O orientador pode, opcionalmente, utilizar a ficha de avaliação presente no ANEXO VI da referida Instrução Normativa e solicitar a assinatura dos membros da banca. No caso da utilização da ficha de avaliação, o orientar fica responsável pelo arquivamento do documento. Para isso, o orientador deverá criar um novo processo SEI e coletar as assinaturas digitais dos membros da banca, não podendo utilizar o processo \seiprocess{} para esse fim.
 	
	A Ata de defesa e o termo de aprovação devem ser entregues pelo orientador por meios do SEI (Sistema Eletrônico de Informações) no processo \seiprocess{} (Graduação: defesa de trabalho de conclusão de curso), da unidade COELE-TD, seguindo os procedimentos presentes na base de conhecimento. \textbf{Caso o orientador não esteja lotado na COELE-TD, este deve pedir acesso ao processo com antecedência para o professor responsável pelo TCC através do e-mail douglascoutinho@utfpr.edu.br}.

	Durante a banca, a ata de defesa e o termo de aprovação devem ser assinadas pelo presidente e avaliadores. Não é necessária a assinatura do candidato. 
	
	\subsection{Base do conhecimento SEI}
	
	Para mais informações sobre a ata de defesa e o termo de aprovação o Orientador pode verificar a base de conhecimento do referido processo, clicando no ícone em formato de B ao lado do número do processo.
	
	\subsection{Divulgação dos resultados}
	
	As notas devem ser divulgadas pelos orientadores logo após a defesa ou após as correções, quando for o caso.

	\subsection{Declarações de participação e orientação}

	A ata de defesa serve como comprovação de participação em banca e orientação de trabalho de TCC. Declarações específicas poderão ser emitidas pelo PRATCC ou pela coordenação de curso mediante justificativa.
	
	\section{Licenciamento}
	
	Os trabalhos de TCC2 do curso seguem, preferencialmente, a licença \textbf{CC BY-NC}, mas outros tipos podem ser utilizadas. As licenças permitidas pela a UTFPR estão listadas na \href{https://sei.utfpr.edu.br/sei/publicacoes/controlador_publicacoes.php?acao=publicacao_visualizar&id_documento=2615190&id_orgao_publicacao=0}{RESOLUÇÃO CONJUNTA COPPG/COGEP Nº 01/2021, DE 10 DE NOVEMBRO DE 2021}. O tipo de licença deve ser escolhido em comum acordo entre o orientador e orientado.
	
	\section{Entrega da versão final de TCC2}
	
	O orientador deverá seguir os procedimentos presentes no site da biblioteca da UTFPR, \href{http://portal.utfpr.edu.br/biblioteca/trabalhos-academicos/docentes/procedimento-de-entrega-graduacao}{clique aqui para abrir}.
	
	Dessa forma, a versão final do TCC2, com as correções exigidas pela banca, junto com os produtos educacionais a ela relacionados, deverá ser entregue pelo orientador por meio da planilha de informações (veja a \autoref{sec:pla}). Na planilha, preencher a parte referente à \textbf{etapa 3} com um link da nuvem da utfpr. O link deverá conter o texto de TCC junto com os produtos educacionais a ela relacionados, e documentações adicionais, quando for o caso.
	
	\textbf{O arquivo deve estar no padrão PDF/A.} Os anexos da \href{https://sei.utfpr.edu.br/sei/publicacoes/controlador_publicacoes.php?acao=publicacao_visualizar&id_documento=2042165&id_orgao_publicacao=0}{INSTRUÇÃO NORMATIVA PROGRAD 1}, de 01 de fevereiro de 2021 devem ser anexados ao mesmo formulário quando for o caso. 

	\subsection{Documentações adicionais}
    Para os TCCs que possuam informações obtidas junto à empresas/organizações/instituições públicas ou privadas, incluindo a própria UTFPR, deve ser preenchido o \href{https://sei.utfpr.edu.br/sei/publicacoes/controlador_publicacoes.php?acao=publicacao_visualizar&id_documento=2651593&id_orgao_publicacao=0}{Termo de Autorização para Divulgação de Informações de Empresas}, que deve ser assinado pelo dirigente máximo, ou o respectivo responsável pelo plano de trabalho, ou a quem for delegada essa responsabilidade na empresa/organização/instituição. O documento original ficará sob a responsabilidade do autor do trabalho a quem compete a apresentação de provas em caso judicial e terá cópia digitalizada arquivada pela Coordenação do Curso.

	Caso a versão final do TCC tenha restrição de acesso ao texto completo decorrente do uso de informações de empresas, caberá ao orientador informar ao PRATCC a data em que o acesso será liberado no Repositório Institucional da UTFPR (RIUT).
		
	Após inserir as informações na planilha a nota do TCC2 será inserida no sistema acadêmico. Caso o orientador não envie a versão final, indicando que as modificações sugeridas pela banca não foram atendidas, a nota atribuída será zero.
\end{document}