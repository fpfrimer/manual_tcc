\documentclass[a4paper, 12pt]{article}

% Pacotes:
\usepackage[brazilian]{babel}
\usepackage[utf8]{inputenc}
\usepackage[T1]{fontenc}
\usepackage[margin=2cm]{geometry}
\usepackage{indentfirst}
\usepackage[colorlinks=true, urlcolor=blue]{hyperref}

% Comandos:
\newcommand{\ordm}{\textsuperscript{\underline{o}}}
\newcommand{\ordf}{\textsuperscript{\underline{a}}}
\newcommand{\utf}{Universidade Tecnológica Federal do Paraná - Câmpus Toledo}
\newcommand{\tcc}{Trabalho de Conclusão de Curso}

\title{Procedimentos de TCC2 - 2022/2\\\textbf{Engenharia Eletrônica}}
\date{***}
\author{Universidade Tecnológica Federal do Paraná - Câmpus Toledo\\Coordenação do curso de Engenharia Eletrônica}


\begin{document}
    \maketitle
    
    O Professor Responsável pela Atividade de \tcc{ } (TCC) de Engenharia Eletrônica da \utf, no uso de suas atribuições estabelecidas pela Resolução N\textsuperscript{o} 018/18 – COEPP, de 11 de abril de 2018, torna público o cronograma de atividades de TCC2 para o primeiro semestre de 2022.

    %\section{Medidas necessárias devido à pandemia do COVID-19}

    %Considerando a pandemia de COVID-19 e a Instrução Normativa N\textsuperscript{o} 08, de 11 de julho de 2020 - UTFPR todas as defesas de TCC de que tratam esse edital devem ser feitas de forma remota através de videoconferência. A ferramenta de reunião digital a ser utilizada é de responsabilidade do(a) Orientador(a).
    
    %O preenchimento de todos os formulários citados pode ser feito de forma digital, com inclusão de assinatura escaneada, ou assinado manualmente para posterior digitalização.

    \section{Modelo de TCC}
    Os trabalhos devem seguir os modelos postados na página do moodle \href{https://moodle.utfpr.edu.br/course/view.php?id=15669}{clique aqui para acessar}.
    
    \section{Cronograma}
    \label{sec:CRO}
    
    Os procedimentos para a disciplina de TCC2 seguirão o seguinte cronograma:
    \begin{itemize}
    	\item \textbf{Início das atividades de TCC2}: 11/08/2022;
    	\item \textbf{Período de agendamento}: Até 19/11/2022 - 20 dias antes do final do semestre (a versão final do texto deve ser entregue para a banca no mesmo prazo);
    	\item \textbf{Período de realização das Bancas}: Até 26/06/2022 - 10 dias antes do final do semestre (o agendamento deve ocorrer com o mínimo de 10 dias de antecedência);
    	\item \textbf{Prazo para a entrega da versão final}: 05/07/2022.
    \end{itemize}

	O atraso no preenchimento de qualquer formulário ou entrega de documentação pode acarretar na não aprovação dos discentes envolvidos no trabalho.    
        
   \section{Agendamento da defesa de TCC2}

    A banca de TCC2 pode ser realizada a qualquer momento até 10 dias antes do final do semestre, sendo de responsabilidade do(a) orientador(a) a escolha de dois avaliadores. Também é de responsabilidade do(a) orientador(a) agendar o dia, horário e local de apresentação. Um dos avaliadores pode ser externo ao corpo docente da UTFPR.
    
    Para formalizar a apresentação de TCC2 o(a) Orientador(a) deverá preencher o Formulário de informe de agendamento de Banca de TCC2: \href{https://forms.gle/QVNraTXAZ27V1xSK8}{Clique aqui para acessar ao formulário}
    
    O formulário deverá ser preenchido com uma antecedência mínima de 10 dias antes da apresentação.
    
    \textbf{O formulário deve ser preenchido única e exclusivamente pelo(a) orientador(a)}.
		
    Observações importantes:
    
    \begin{itemize}
    	\item A entrega do texto para a Banca deve ser feita com antecedência mínima de 10 (dez) dias antes da defesa;
    	\item Os discentes ou os orientadores são responsáveis por entregar a cópia digital ou impressa do TCC para a banca;
    	%\item A data de apresentação informada no formulário 9 devem ser previamente agendada com a banca pelo(a) Orientador(a);    	
    	\item Mudanças relativas à data de apresentação e a composição dos membros da banca deverão ser informadas e justificadas pelo(a) orientador(a);
    	\item O período de agendamento para os trabalhos de TCC2 está exposto na Seção \ref{sec:CRO};
    \end{itemize}

	\section{Assinaturas externas}
	
	Assinatura externa é aquela feita por pessoas de fora da instituição. No caso do TCC, essas assinaturas são de membros externos da banca. Para poder assinar a documentação os membros externos devem estar cadastrados no SEI. O cadastramento deve ser feito pelo(a) orientador(a), que deverá entrar em contato com um facilitador do SEI no campus. Pergunte ao \href{http://portal.utfpr.edu.br/servidores/servicos-servidor/sei/facilitadores}{facilitador} quais dados são necessários para cadastrar a pessoa.

	\section{O seminário de TCC2}
	
	O(a) presidente da banca é o orientador(a), que deverá coordenar as etapas de apresentação: defesa, arguição e deliberação. 
	
	Os candidatos terão entre 15 e 20 minutos para realizar a defesa. Logo após, cada membro da banca terá de 15 a 20 minutos para apresentar suas correções e questionamentos. Por fim, os membros da banca devem se reunir de forma reservada para deliberar o resultado.
	
	%Cada membro da banca deve receber um cópia digital da Ficha de Avaliação de TCC 2 (Formulário 10). O preenchimento pode ser feito de forma digital, com inclusão de assinatura escaneada, ou assinado manualmente para posterior digitalização. \textbf{As fichas preenchidas não devem ser apresentadas aos candidatos}.
	
	%O(a) orientador(a) deve entregar as fichas de avaliação preenchidas para o professor responsável através do e-mail pfrimer@utfpr.edu.br.
	
	A nota final será composta por: 20\% da nota do orientador e 40\% de cada membro. O resultado da média ponderada deve ser inserida na ATA de defesa.
	
	A Ata de defesa e o termo de aprovação devem ser entregues pelo orientador através do SEI (Sistema Eletrônico de Informações) no processo 23064.008580/2022-58 (Graduação: defesa de trabalho de conclusão de curso), da unidade COELE-TD, seguindo os procedimentos presentes na base de conhecimento. \textbf{Caso o(a) orientador(a) não esteja lotado na COELE-TD, este deve pedir acesso ao processo com antecedência para o professor responsável pelo TCC através do e-mail prat-coele-td@utfpr.edu.br.}
	
	\subsection{Base do conhecimento SEI}
	
	Para mais informações o(a) Orientador(a) pode verificar a base de conhecimento do referido processo, clicando no ícone em formato de B ao lado do número do processo.
	
	\subsection{Critérios de avaliação}
	
	O(a) orientador(a) deve considerar os seguintes critérios de avaliação:
	
	\begin{itemize}
		\item Comprometimento com as determinações da orientação e participação das reuniões;
		\item Demonstrou interesse pelo aprendizado e pelo projeto;
		\item Demonstrou dedicação ao escrever as versões entregues para a revisão;
		\item Demonstrou autonomia para encontrar soluções para o projeto;
		\\item Qualidade geral do texto final.
	\end{itemize}

	Cada membro da banca deve considerar os seguintes critérios de avaliação:
	
	\begin{itemize}
		\item Qualidade geral do texto final;
		\item Demonstrou conhecimento sobre o assunto apresentado;
		\item Viabilidade e exequibilidade do projeto;
		\item Relevância na área do curso.
	\end{itemize}

	Os critérios de avaliação devem ter pesos iguais.

	\textbf{É de competência do(a) Orientador(a) apresentar os critérios de avaliação para os membros da banca.}	
	
	\section{Divulgação dos resultados}
	
	As notas devem ser divulgadas pelos orientadores logo após a defesa.
	
	\section{Licenciamento}
	
	Os trabalhos de TCC2 do curso seguem, preferencialmente, a licença \textbf{CC BY-NC}. As licenças permitidas pela a UTFPR estão listadas na \href{https://sei.utfpr.edu.br/sei/publicacoes/controlador_publicacoes.php?acao=publicacao_visualizar&id_documento=2615190&id_orgao_publicacao=0}{RESOLUÇÃO CONJUNTA COPPG/COGEP Nº 01/2021, DE 10 DE NOVEMBRO DE 2021}.
	
	\section{Entrega da versão final de TCC2}
	
	O orientador deverá seguir os procedimentos presentes no seguinte link: \href{http://portal.utfpr.edu.br/biblioteca/trabalhos-academicos/docentes/procedimento-de-entrega-graduacao}{clique aqui}.
	
	Dessa forma, a versão final do TCC2, com as correções exigidas pela banca, junto com os produtos educacionais a ela relacionados, deverá ser entregue pelo(a) orientador(a) através do \href{https://forms.gle/LScHkAtDPReTjki46}{formulário de entrega de versão final}. 
	
	O arquivo deve estar no padrão PDF/A. Os anexos da \href{https://sei.utfpr.edu.br/sei/publicacoes/controlador_publicacoes.php?acao=publicacao_visualizar&id_documento=2042165&id_orgao_publicacao=0}{INSTRUÇÃO NORMATIVA PROGRAD 1}, de 01 de fevereiro de 2021 devem ser anexados ao mesmo formulário quando for o caso. 
		
	Após responder ao formulário a nota do TCC2 será inserida no sistema acadêmico. Caso o(a) orientador(a) não envie a versão final, indicando que as modificações sugeridas pela banca não foram atendidas, a nota atribuída será zero.	
		
	\section{Informações adicionais importantes}
	
	Para os TCCs que possuam informações obtidas junto à empresas/organizações/instituições públicas ou privadas, incluindo a própria UTFPR, deve ser preenchido o \href{https://sei.utfpr.edu.br/sei/publicacoes/controlador_publicacoes.php?acao=publicacao_visualizar&id_documento=2651593&id_orgao_publicacao=0}{Termo de Autorização para Divulgação de Informações de Empresas}, que deve ser assinado pelo dirigente máximo, ou o respectivo responsável pelo plano de trabalho, ou a quem for delegada essa responsabilidade na empresa/organização/instituição. O documento original ficará sob a responsabilidade do autor do trabalho a quem compete a apresentação de provas em caso judicial e terá cópia digitalizada arquivada pela Coordenação do Curso.

	Caso a versão final do TCC tenha restrição de acesso ao texto completo decorrente do uso de informações de empresas, caberá ao orientador informar ao PRATCC a data em que o acesso será liberado no Repositório Institucional da UTFPR (RIUT).
	
	
	\section{Casos omissos}
	
	Casos omissos à esse edital serão decididos pela coordenação do curso em conjunto com o professor responsável pela atividade de TCC.

\end{document}