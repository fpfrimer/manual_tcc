\documentclass[a4paper, 12pt]{article}

% Pacotes:
\usepackage[brazilian]{babel}
\usepackage[utf8]{inputenc}
\usepackage[T1]{fontenc}
\usepackage[margin=2cm]{geometry}
\usepackage{indentfirst}
\usepackage[colorlinks=true, urlcolor=blue]{hyperref}

% Comandos:
\newcommand{\ordm}{\textsuperscript{\underline{o}}}
\newcommand{\ordf}{\textsuperscript{\underline{a}}}
\newcommand{\utf}{Universidade Tecnológica Federal do Paraná - Câmpus Toledo}
\newcommand{\tcc}{Trabalho de Conclusão de Curso}

\title{Procedimentos de TCC2 - 2022/2\\\textbf{Engenharia Eletrônica}}
\date{***}
\author{Universidade Tecnológica Federal do Paraná - Câmpus Toledo\\Coordenação do curso de Engenharia Eletrônica}


% Atualizar as datas a cada semestre:
\newcommand{\startdate}{11/08/2022}		% Data de início do semestre
\newcommand{\agendadate}{19/11/2022}	% Prazo para agendamento da banca de TCC
\newcommand{\bancadate}{15/12/2022}		% Prazo para realização das Bancas
\newcommand{\finaldate}{17/12/2022}		% Prazo para a entrega da versão final

\begin{document}
    \maketitle
    
    \section{Modelo de TCC}
    Os trabalhos devem seguir os modelos postados na página da biblioteca \href{http://portal.utfpr.edu.br/biblioteca/trabalhos-academicos}{(clique aqui para acessar)}.
    
    \section{Cronograma}
    \label{sec:CRO}
    
    Os procedimentos para a disciplina de TCC2 seguirão o seguinte cronograma:
    \begin{itemize}
    	\item \textbf{Início das atividades de TCC2}: \startdate;
    	\item \textbf{Prazo para agendamento da banca de TCC}: Até \agendadate;
    	\item \textbf{Prazo para realização das Bancas}: Até \bancadate;
    	\item \textbf{Prazo para a entrega da versão final}: Até \finaldate.
    \end{itemize}

	O atraso da entrega das documentações solicitadas podem acarretar na não aprovação dos discentes envolvidos no trabalho.    
        
   \section{Agendamento da defesa de TCC2}

    A banca de TCC2 pode ser realizada a qualquer momento entre os dias \textbf{\startdate{} e \bancadate{}}, sendo de responsabilidade do(a) orientador(a) a escolha de dois avaliadores. A critério do(a) Orientador(a), um dos membros pode ser externo ao quadro efetivo de docentes da UTFPR, desde que atue em área relevante para o trabalho avaliado. Também é de responsabilidade do(a) orientador(a) agendar o dia, horário e local de apresentação. 
    
    Para formalizar a apresentação de TCC2 o(a) orientador(a) deverá informar o agendamento de Banca de TCC2 para o professor responsável pela atividade de TCC (PRATCC) através de formulário digital: \href{https://forms.gle/QVNraTXAZ27V1xSK8}{Clique aqui para acessar ao formulário}.
    
    O formulário deverá ser preenchido com \textbf{antecedência mínima de 10 dias antes da apresentação}. \textbf{O formulário deve ser preenchido única e exclusivamente pelo(a) orientador(a) ou coorientador(a)}.    
		
    Observações importantes:
    
    \begin{itemize}
    	\item A entrega do texto para a Banca deve ser feita com antecedência mínima de 10 (dez) dias antes da defesa;
    	\item Os discentes ou os orientadores são responsáveis por entregar a cópia digital ou impressa do texto de TCC para a banca;
    	%\item A data de apresentação informada no formulário 9 devem ser previamente agendada com a banca pelo(a) Orientador(a);    	
    	\item Mudanças relativas à data de apresentação e a composição dos membros da banca deverão ser informadas e justificadas pelo(a) orientador(a) ao PRATCC pelo e-mail prat-coele-td@utfpr.edu.br;
    	\item O período de agendamento para os trabalhos de TCC2 está exposto na Seção \ref{sec:CRO};
    \end{itemize}

	\section{Assinaturas para avaliadores externos}
	
	Assinatura externa é aquela feita por pessoas de fora da instituição. No caso do TCC, essas assinaturas são de membros externos da banca. Para poder assinar a documentação os membros externos devem estar cadastrados no SEI. O cadastramento deve ser feito \textbf{pelo(a) orientador(a)}, que deverá entrar em contato com um facilitador do SEI no campus. Pergunte ao facilitador do SEI no campus Toledo\href{http://portal.utfpr.edu.br/servidores/servicos-servidor/sei/facilitadores}{(clique aqui para a lista)} quais procedimentos e dados são necessários para cadastrar o(a) avaliador(a).

	\section{O seminário de TCC2}
	
	O(A) presidente da banca é o orientador(a) ou coorientador(a), que deverá coordenar as etapas de apresentação: defesa, arguição e deliberação. 
	
	Os candidatos terão 15 min para realizar a defesa. Logo após, cada membro da banca terá até 15 minutos para apresentar suas correções e questionamentos. Por fim, os membros da banca devem se reunir de forma reservada para deliberar o resultado. O rito da banca deve ter duração máxima de 1 h.
	
	Durante a deliberação, os os membros da banca devem preencher o formulário de avaliação (\href{https://nuvem.utfpr.edu.br/index.php/s/gK76empfxjAIMbl}{clique aqui para baixar}). O formulário irá calcular a \textbf{nota final} quando a nota de todos os critérios forem inseridos. A nota final será composta por: 20\% da nota do orientador e 40\% de cada membro. O resultado da média ponderada deve ser inserida na ATA de defesa. 
	
	%Cada membro da banca deve receber um cópia digital da Ficha de Avaliação de TCC 2 (Formulário 10). O preenchimento pode ser feito de forma digital, com inclusão de assinatura escaneada, ou assinado manualmente para posterior digitalização. \textbf{As fichas preenchidas não devem ser apresentadas aos candidatos}.
	
	%O(a) orientador(a) deve entregar as fichas de avaliação preenchidas para o professor responsável através do e-mail pfrimer@utfpr.edu.br.	
	
	A Ata de defesa e o termo de aprovação devem ser entregues pelo orientador através do SEI (Sistema Eletrônico de Informações) no processo 23064.008580/2022-58 (Graduação: defesa de trabalho de conclusão de curso), da unidade COELE-TD, seguindo os procedimentos presentes na base de conhecimento. \textbf{Caso o(a) orientador(a) não esteja lotado na COELE-TD, este deve pedir acesso ao processo com antecedência para o professor responsável pelo TCC através do e-mail prat-coele-td@utfpr.edu.br}.

	Durante a banca, a ata de defesa e o termo de aprovação devem ser assinadas pelo presidente e avaliadores. Não é necessária a assinatura do candidato. 
	
	\subsection{Base do conhecimento SEI}
	
	Para mais informações sobre a ata de defesa e o termo de aprovação o(a) Orientador(a) pode verificar a base de conhecimento do referido processo, clicando no ícone em formato de B ao lado do número do processo.
	
	\subsection{Critérios de avaliação}
	
	O(a) orientador(a) deve considerar os seguintes critérios de avaliação:
	
	\begin{itemize}
		\item \textbf{(Peso 15)} Comprometimento com as determinações da orientação e participação das reuniões;
		\item \textbf{(Peso 15)} Demonstrou interesse pelo aprendizado e pelo projeto;
		\item \textbf{(Peso 15)} Demonstrou dedicação ao escrever as versões entregues para a revisão;
		\item \textbf{(Peso 15)} Demonstrou autonomia para encontrar soluções para o projeto;
		\item \textbf{(Peso 40)} Qualidade geral do texto final (normas da ABNT, clareza do texto, ortografia, coesão, entre outros).
	\end{itemize}

	Cada membro da banca deve considerar os seguintes critérios de avaliação:
	
	\begin{itemize}
		\item \textbf{(Peso 20)} Demonstrou conhecimento sobre o assunto durante a apresentação;
		\item \textbf{(Peso 25)} Qualidade da Discussão do Resultados e conclusões;
		\item \textbf{(Peso 25)} Descreve de forma clara a metodologia / A metodologia utilizada foi adequada;
		\item \textbf{(Peso 30)} Qualidade geral do texto final (normas da ABNT, clareza do texto, ortografia, coesão, entre outros).
	\end{itemize}

	\section{Divulgação dos resultados}
	
	As notas devem ser divulgadas pelos orientadores logo após a defesa.
	
	\section{Licenciamento}
	
	Os trabalhos de TCC2 do curso seguem, preferencialmente, a licença \textbf{CC BY-NC}, mas outros tipos podem ser utilizadas. As licenças permitidas pela a UTFPR estão listadas na \href{https://sei.utfpr.edu.br/sei/publicacoes/controlador_publicacoes.php?acao=publicacao_visualizar&id_documento=2615190&id_orgao_publicacao=0}{RESOLUÇÃO CONJUNTA COPPG/COGEP Nº 01/2021, DE 10 DE NOVEMBRO DE 2021}. O tipo de licença deve ser escolhido em comum acordo entre o(a) orientador(a) e orientado(a).
	
	\section{Entrega da versão final de TCC2}
	
	O orientador deverá seguir os procedimentos presentes no site da biblioteca da UTFPR, \href{http://portal.utfpr.edu.br/biblioteca/trabalhos-academicos/docentes/procedimento-de-entrega-graduacao}{clique aqui para abrir}.
	
	Dessa forma, a versão final do TCC2, com as correções exigidas pela banca, junto com os produtos educacionais a ela relacionados, deverá ser entregue pelo(a) orientador(a) através do \href{https://forms.gle/LScHkAtDPReTjki46}{formulário de entrega de versão final}. 
	
	\textbf{O arquivo deve estar no padrão PDF/A.} Os anexos da \href{https://sei.utfpr.edu.br/sei/publicacoes/controlador_publicacoes.php?acao=publicacao_visualizar&id_documento=2042165&id_orgao_publicacao=0}{INSTRUÇÃO NORMATIVA PROGRAD 1}, de 01 de fevereiro de 2021 devem ser anexados ao mesmo formulário quando for o caso. 
		
	Após responder ao formulário a nota do TCC2 será inserida no sistema acadêmico. Caso o(a) orientador(a) não envie a versão final, indicando que as modificações sugeridas pela banca não foram atendidas, a nota atribuída será zero.	
		
	\section{Informações adicionais importantes}
	
	Para os TCCs que possuam informações obtidas junto à empresas/organizações/instituições públicas ou privadas, incluindo a própria UTFPR, deve ser preenchido o \href{https://sei.utfpr.edu.br/sei/publicacoes/controlador_publicacoes.php?acao=publicacao_visualizar&id_documento=2651593&id_orgao_publicacao=0}{Termo de Autorização para Divulgação de Informações de Empresas}, que deve ser assinado pelo dirigente máximo, ou o respectivo responsável pelo plano de trabalho, ou a quem for delegada essa responsabilidade na empresa/organização/instituição. O documento original ficará sob a responsabilidade do autor do trabalho a quem compete a apresentação de provas em caso judicial e terá cópia digitalizada arquivada pela Coordenação do Curso.

	Caso a versão final do TCC tenha restrição de acesso ao texto completo decorrente do uso de informações de empresas, caberá ao orientador informar ao PRATCC a data em que o acesso será liberado no Repositório Institucional da UTFPR (RIUT).
	
	

\end{document}