\documentclass[a4paper, 12pt]{article}

% Pacotes:
\usepackage[brazilian]{babel}
\usepackage[utf8]{inputenc}
\usepackage[T1]{fontenc}
\usepackage[margin=2cm]{geometry}
\usepackage{indentfirst}
\usepackage[colorlinks=true, urlcolor=blue]{hyperref}
\usepackage{tcolorbox}

% Comandos:
\newcommand{\ordm}{\textsuperscript{\underline{o}}}
\newcommand{\ordf}{\textsuperscript{\underline{a}}}
\newcommand{\utf}{Universidade Tecnológica Federal do Paraná - Câmpus Toledo}
\newcommand{\tcc}{Trabalho de Conclusão de Curso}

\title{Procedimentos de TCC1 - 2024/2\\\textbf{Engenharia Eletrônica}}
\date{***}
\author{Universidade Tecnológica Federal do Paraná - Câmpus Toledo\\Coordenação do curso de Engenharia Eletrônica}


% Atualizar as datas a cada semestre:
\newcommand{\startdate}{01/10/2024}		% Data de início do semestre
\newcommand{\agendadate}{08/02/2025}	% Prazo para desenvolvimento e escrita do TCC
\newcommand{\bancadate}{15/02/2025}		% Prazo para realização das Bancas
\newcommand{\finaldate}{22/02/2025}		% Prazo para a entrega da versão final
\newcommand{\seiprocess}{23064.008517/2024-83} % Processo para as ATAs

\begin{document}
    \maketitle
    
    \section{Modelo de TCC}
    %Sugere-se que os projetos devam seguir os modelos postados na página da biblioteca \href{http://portal.utfpr.edu.br/biblioteca/trabalhos-academicos}{(clique aqui para acessar)}, fazendo-se as devidas modificações, pois tais textos \textbf{não serão publicados}. Existem modelos em docx e \LaTeX{}. Recomenda-se a leitura da \href{https://sei.utfpr.edu.br/sei/publicacoes/controlador_publicacoes.php?acao=publicacao_visualizar&id_documento=2615190&id_orgao_publicacao=0}{Resolução Conjunta COPPG/COGEP n\textordmasculine{} 01/2021}, de 10 de novembro de 2021, que dispõe sobre a Política de licenciamento das versões finais dos Trabalhos de Conclusão de Curso da Graduação (TCC) e dos Programas de Pós-Graduação Stricto Sensu (dissertações e teses), bem como dos produtos educacionais e tecnológicos a elas vinculados, produzidas no âmbito da Universidade Tecnológica Federal do Paraná, pois tais procedimentos deverão ser seguidos no desenvolvimento do TCC2.

	% Melhorado com IA:
	Para a elaboração dos projetos de TCCs, é recomendado utilizar os modelos disponíveis na página da biblioteca da UTFPR \href{http://portal.utfpr.edu.br/biblioteca/trabalhos-academicos}{(clique aqui para acessar)}. Esses modelos, disponíveis em formatos docx e \LaTeX, devem ser adaptados conforme necessário, já que os textos de TCC1 não são destinados à publicação. %Além disso, é essencial consultar a \href{https://sei.utfpr.edu.br/sei/publicacoes/controlador_publicacoes.php?acao=publicacao_visualizar&id_documento=2615190&id_orgao_publicacao=0}{Resolução Conjunta COPPG/COGEP nº 01/2021}, datada de 10 de novembro de 2021, que estabelece a política de licenciamento para as versões finais dos TCCs de graduação e pós-graduação stricto sensu, incluindo dissertações, teses, e produtos educacionais e tecnológicos relacionados, desenvolvidos na UTFPR. Este documento orienta os procedimentos a serem seguidos no desenvolvimento do TCC2.

	%Alternativamente, no caso da escrita em \LaTeX{}, recomenda-se a utilização do modelo do curso devido a facilidade de uso e maior compatibilidade com pacotes atualizados. O modelo do curso pode ser baixado em \url{https://github.com/coele-td/TCCmodelv3}.
    
	%Melhorado com IA
	Para a elaboração de TCCs em \LaTeX{}, recomenda-se o uso do modelo específico do curso, que oferece facilidade de uso e compatibilidade com os pacotes mais recentes. Este modelo está disponível para download no seguinte endereço: \url{https://github.com/coele-td/TCCmodelv3}. Utilizar este modelo pode ajudar a garantir que a formatação e estrutura do seu trabalho estejam alinhadas com as expectativas do curso.


    \section{Cronograma}
    \label{sec:CRO}
    
    Os procedimentos para a disciplina de TCC1 seguirão o seguinte cronograma:
    \begin{itemize}
    	\item \textbf{Etapa 1} - Desenvolvimento do TCC1: \startdate{} à \agendadate;
    	\item \textbf{Etapa 2} - Agendamento  e realização das bancas (até \bancadate);
    	\item \textbf{Etapa 3} - Entrega da versão final para o orientador: até \finaldate.    	
    \end{itemize}

	A entrega das documentações solicitadas é requisito para a aprovação em TCC1.

	\section{Planilha de informações}
	\label{sec:pla}
	
	Trata-se de uma planilha compartilhada onde apenas os orientadores podem inserir as informações solicitadas em cada etapa do cronograma. As informações necessárias são descritas nas próximas seções. A planilha possui acesso restrito. Portanto, orientadores devem solicitar acesso assim que acessarem o link pela primeira vez.

	%Esta seção introduz uma planilha compartilhada, crucial para a organização e acompanhamento do TCC1. Nela, orientadores e candidatos devem inserir as informações solicitadas em cada etapa do cronograma. As informações necessárias são descritas nas próximas seções. Como a planilha possui acesso restrito, orientadores e candidatos devem solicitar acesso assim que acessarem o link pela primeira vez, que será concedido em até 5 dias úteis.
	
	O link para editar a planilha pode ser acessado \href{https://docs.google.com/spreadsheets/d/1AfsG1P6wrw42CbZCDbQzsXFXmxfrMmZ7gpn8NvR8cq4/edit?usp=sharing}{clicando aqui}.
	
	Adicionalmente, a planilha pode ser visualizada \href{https://docs.google.com/spreadsheets/d/e/2PACX-1vQKQ_bI1Ka_BKbx_LZkVxxYrV69ROWFWqE8V_niYUJP7EjumnNhUP65tGiUxh32qgDYW_WWdC5nJelJ/pubhtml}{clicando aqui}. Esse link será publicado na página do curso.

	É importante salientar que toda edição feita na planilha, assim como seu histórico, é monitorada pelo PRATCC. Boas condutas incluem não alterar dados de outros projetos sem autorização, utilizar a planilha de maneira responsável e verificar duplamente as informações antes de salvá-las. Violações dessas diretrizes podem resultar em advertências formais ou, em casos graves, reprovação na disciplina de TCC1. É fundamental que cada participante contribua para a integridade e precisão dos dados, mantendo um ambiente colaborativo e respeitoso.

	\section{Etapa 1 - Desenvolvimento do TCC1}
	\label{sec:et1}
	Na etapa 1, os candidatos escrevem o projeto de TCC1. Recomenda-se uma reunião de orientação por semana. \textbf{Nesta etapa, é necessário inserir os dados da Etapa 1 na planilha de informações} (veja a \autoref{sec:pla}).
	
	Adicionalmente, os orientadores devem inserir no processo SEI \seiprocess{} os termos de orientação, conforme o seguinte modelo:

	\begin{tcolorbox}[title=Termo de Orientação de TCC]
		\textbf{Termo de Compromisso de Orientação de TCC}
		
		Eu, \underline{\hspace{5cm}}, como orientador(a), juntamente com os coorientadores (opcional) \underline{\hspace{5cm}}, comprometo-me a orientar o(a) aluno(a) ou grupo de alunos \underline{\hspace{5cm}}, na elaboração do Trabalho de Conclusão de Curso (TCC), nas disciplinas de TCC1 e TCC2, conforme as normas e diretrizes estabelecidas pela instituição.
		
		\textbf{Compromissos do Orientador:}
		\begin{itemize}
			\item Orientar o aluno na escolha do tema, definição dos objetivos, metodologia de pesquisa, e elaboração do trabalho.
			\item Acompanhar o desenvolvimento do projeto, oferecendo suporte técnico e acadêmico necessário.
			\item Avaliar periodicamente o progresso do trabalho e fornecer feedback construtivo.
		\end{itemize}
		
		\textbf{Compromissos do Aluno:}
		\begin{itemize}
			\item Cumprir as etapas do trabalho dentro dos prazos estabelecidos.
			\item Manter comunicação regular com o orientador e coorientador, se aplicável.
			\item Seguir as orientações e correções sugeridas.
		\end{itemize}
		
		\textbf{Data:} \underline{\hspace{2cm}}/\underline{\hspace{2cm}}/\underline{\hspace{4cm}}
		
		\textbf{Assinatura do Orientador:} \underline{\hspace{6cm}}
		
		\textbf{Assinatura do(s) Coorientador(es):} \underline{\hspace{6cm}} (opcional)
		
		\textbf{Assinatura do(s) Aluno(s):} \underline{\hspace{6cm}}
	\end{tcolorbox}

	Para isso, utilize o documento tipo ``Termo - '', Entre com o nome do aluno no campo ``Número''. Existirá um modelo do documento no processo.


    \section{Etapa 2 - Agendamento e realização das bancas}
	\label{sec:et2}

	Esta etapa é dedicada ao agendamento e à execução das sessões de defesa do TCC, gerenciadas pelo orientador. Durante este período, é necessário inserir os dados da seção ``Etapa 2'' na planilha de informações com os detalhes pertinentes ao agendamento e às datas das bancas. Este procedimento assegura a organização e a comunicação efetiva entre todas as partes envolvidas, facilitando a coordenação dos eventos.
   	
	\subsection{Agendamento da defesa de TCC1}

    A banca de TCC1 pode ser realizada a qualquer momento entre os dias \textbf{\startdate{} e \bancadate{}}, sendo de responsabilidade do orientador a escolha de até dois avaliadores. A critério do Orientador, um dos membros pode ser externo ao quadro efetivo de docentes da UTFPR, desde que atue em área relevante para o trabalho avaliado. Também é de responsabilidade do orientador agendar o dia, horário e local de apresentação. O local de apresentação deve ser reservado pelo orientador.

	A banca é formada por:

	\begin{itemize}
		\item Orientador (ou coorientador, se houver);
		\item Professor que atua na área de eletrônica;
		\item (opcional) Professor, pesquisador ou profissional graduado que atua em área relevante para o projeto de TCC1,  podendo fazer parte ou não do quadro efetivo ou contratado da UTFPR.
	\end{itemize}
    
    Para formalizar a apresentação de TCC1 o orientador deverá informar o agendamento de Banca para o professor responsável pela atividade de TCC (PRATCC) por meio da planilha de informações (veja a \autoref{sec:pla}), na região indicada como Etapa 2.
    
    A planilha deverá ser preenchida com \textbf{antecedência mínima de 7 dias antes da apresentação}.    
		
    Observações importantes:
    
    \begin{itemize}
    	\item A entrega do texto para a Banca deve ser feita com antecedência mínima de 7 dias antes da defesa;
    	\item Os discentes ou os orientadores são responsáveis por entregar a cópia digital ou impressa do texto de TCC para a banca;
    	%\item A data de apresentação informada no formulário 9 devem ser previamente agendada com a banca pelo(a) Orientador(a);    	
    	\item Mudanças relativas à data de apresentação e a composição dos membros da banca deverão ser informadas e justificadas pelo(a) orientador(a) ao PRATCC pelo e-mail tcccoeleutfprtd@gmail.com;
    	\item O período de agendamento para os trabalhos de TCC1 está exposto na Seção \ref{sec:CRO};
    \end{itemize}

	\section{Assinaturas para avaliadores externos}
	
	Assinatura externa é aquela feita por pessoas de fora da instituição. No caso do TCC, essas assinaturas são de membros externos da banca. Para poder assinar a documentação os membros externos devem estar cadastrados no SEI. O cadastramento deve ser feito \textbf{pelo orientador}, que deverá entrar em contato com um facilitador do SEI no campus. Pergunte ao facilitador do SEI no campus Toledo \href{http://portal.utfpr.edu.br/servidores/servicos-servidor/sei/facilitadores}{(clique aqui para a lista)} quais procedimentos e dados são necessários para cadastrar o avaliador.

	\subsection{O seminário de TCC1}
	
	O presidente da banca é o orientador ou coorientador, que deverá coordenar as etapas de apresentação: defesa, arguição e deliberação. 
	
	Os candidatos terão 10 min para realizar a defesa. Logo após, cada membro da banca terá até 10 minutos para apresentar suas correções e questionamentos. Por fim, os membros da banca devem se reunir de forma reservada para deliberar o resultado. O rito da banca deve ter duração máxima de 45 min.	

	\subsection{Critérios de avaliação}

	Durante a deliberação, o orientador deve avaliar de acordo com os seguintes critérios:

	\begin{itemize}
		\item Comprometimento com as determinações da orientação e participação das reuniões (peso de 15\%);
		\item Demonstrou interesse pelo aprendizado e pelo projeto (peso de 15\%);
		\item Demonstrou dedicação ao escrever as versões entregues para a revisão (peso de 20\%);
		\item Demonstrou autonomia para encontrar soluções para o projeto (peso de 20\%);
		\item Qualidade geral do texto final (normas da ABNT, clareza do texto, ortografia, coesão, entre outros) (peso de 30\%);
	\end{itemize}

	Durante a deliberação, os membros devem avaliar de acordo com os seguintes critérios:

	\begin{itemize}
		\item Demonstrou conhecimento sobre o assunto durante a apresentação (peso de 20\%);
		\item Viabilidade e exequibilidade do projeto (peso de 25\%);
		\item Relevância do projeto na área do curso (acadêmico, vínculo com o ambiente de atuação profissional previsto no perfil do egresso e abordagem inovadora) (peso de 25\%);
		\item Qualidade geral do texto final (normas da ABNT, clareza do texto, ortografia, coesão, entre outros) (peso de 30\%);
	\end{itemize}

 	Esse critérios deverão ser divulgados pelo orientador ou candidatos no ato de entrega do texto de TCC1. A nota deve ser calculada de acordo com a \href{https://sei.utfpr.edu.br/sei/publicacoes/controlador_publicacoes.php?acao=publicacao_visualizar&id_documento=3053252&id_orgao_publicacao=0}{instrução Normativa COELE-TD/UTFPR nº 5}, de  13 de junho de 2022. 
	
	O orientador pode, opcionalmente, utilizar a ficha de avaliação presente no ANEXO IV da referida Instrução Normativa e solicitar a assinatura dos membros da banca. No caso da utilização da ficha de avaliação, o orientar fica responsável pelo arquivamento do documento. Para isso, o orientador deverá criar um novo processo SEI e coletar as assinaturas digitais dos membros da banca, não podendo utilizar o processo \seiprocess{} para esse fim.

	A Ata de defesa de TCC1 deve ser entregue pelo orientador por meio do SEI (Sistema Eletrônico de Informações) no processo \seiprocess{} (o mesmo processo dos termos de orientação), da unidade COELE-TD, conforme o modelo presente no processo. \textbf{Caso o(a) orientador(a) não esteja lotado na COELE-TD, este deve pedir acesso ao processo com antecedência para o professor responsável pelo TCC através do e-mail tcccoeleutfprtd@gmail.com}.

	\textbf{Atenção: Atualizar a ata com o nome correto da disciplina, trabalho de conclusão de curso 1 - TCC1}.

	Durante a banca, a ata de defesa deve ser assinado pelo presidente e avaliadores. Não é necessária a assinatura do candidato.	
	
	\subsection{Divulgação dos resultados}
	
	As notas devem ser divulgadas pelos orientadores logo após a defesa ou após as correções, quando for o caso.

	\subsection{Declarações de participação e orientação}

	A ata de defesa serve como comprovação de participação em banca e orientação de trabalho de TCC1. Declarações específicas poderão ser emitidas pelo PRATCC ou pela coordenação de curso mediante justificativa.	
		
	\section{Etapa 3 - Entrega da versão final de TCC1}	
		
	A versão final do TCC1, com as correções exigidas pela banca, deverá ser entregue pelo orientador por meio da planilha de informações (veja a \autoref{sec:pla}). Na planilha, preencher a parte referente à \textbf{etapa 3} com um link da nuvem da utfpr. O link deverá conter o texto de TCC junto documentações adicionais, quando for o caso.	
			
	Após inserir as informações na planilha a nota do TCC1 será inserida no sistema acadêmico. Caso o orientador não envie a versão final, indicando que as modificações sugeridas pela banca não foram atendidas, a nota atribuída será 5,9, o que acarreta na reprovação do candidato.

\end{document}